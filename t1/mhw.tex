
\documentclass[titlepage]{article}
\usepackage{amsmath}
\usepackage{fancyhdr}
\usepackage{graphicx}
\usepackage[T1, T2A]{fontenc}
\usepackage[utf8]{inputenc}
\usepackage[russian, english]{babel}


\title{Линейная алгебра. Типовичок}
\date{\today}
\author{Юрий Баринов}

\pagestyle{fancy}
\lhead{Юрий Баринов}\chead{Линейная алгебра. Типовичок}\rhead{\today}

\newcounter{ProblemNum}

\newcommand{\problem}[1]{
\stepcounter{ProblemNum}
\newpage
\section*{Задача \theProblemNum. #1}
}

\newcommand{\solution}{
	\subsection*{Решение}
}

\begin{document}
\maketitle{}

\problem{}


\begin{enumerate}
	\item Матрицу 2A-3B
	\item Произведение матриц AB и BA.\\Выяснить, являются ли данные матрицы перестановочными
	\item Определитель матрицы A
	\item Матрицу, обратную B. Выполнить проверку.
\end{enumerate}

\begin{equation*}
A =	
\begin{bmatrix}
	5 & 3 & -7 \\
	-1 & 6 & -3 \\
	2 & -2 & 1
\end{bmatrix}
,
B =	
\begin{bmatrix}
	4 & -1 & 3 \\
	4 & -2 & -6 \\
	2 & 0 & 3
\end{bmatrix}
\end{equation*}
\solution{}
\begin{equation}
	2A - 3B =
2
\begin{bmatrix}
	5 & 3 & -7 \\
	-1 & 6 & -3 \\
	2 & -2 & 1
\end{bmatrix}
-
3
\begin{bmatrix}
	4 & -1 & 3 \\
	4 & -2 & -6 \\
	2 & 0 & 3
\end{bmatrix}
=
\begin{bmatrix}
	-2 & 9 & -23 \\
	-14 & 18 & 12 \\
	-2 & -4 & -11
\end{bmatrix}
\end{equation}


\begin{equation}
	AB =
\begin{bmatrix}
	5 & 3 & -7 \\
	-1 & 6 & -3 \\
	2 & -2 & 1
\end{bmatrix}
*
\begin{bmatrix}
	4 & -1 & 3 \\
	4 & -2 & -6 \\
	2 & 0 & 3
\end{bmatrix}
\end{equation}
\begin{equation}
AB
=
\begin{bmatrix}
	20 + 12 -14 & -5 - 6 & 15 - 18 - 21 \\
	-4 + 24 - 6 & 1 - 12 & -3 - 36 - 9 \\
	8 - 8 + 2 & -2 + 4 & 6 + 12 + 3
\end{bmatrix}
=
\begin{bmatrix}
	18 & -11 & -24  \\
	14 & -11 & -48  \\
	2 & 2 & 21
\end{bmatrix}
\end{equation}

\begin{equation}
	BA =
\begin{bmatrix}
	4 & -1 & 3 \\
	4 & -2 & -6 \\
	2 & 0 & 3
\end{bmatrix}
*
\begin{bmatrix}
	5 & 3 & -7 \\
	-1 & 6 & -3 \\
	2 & -2 & 1
\end{bmatrix}
\end{equation}
\begin{equation}
AB
=
\begin{bmatrix}
	20 + 1 + 6 & 12 - 6 - 6 & -28 + 3 + 3 \\
	20 + 2 - 12 & 12 - 12 + 12 & -28 + 6 - 6 \\
	10 + 6 & 6 - 6 & -14 + 3
\end{bmatrix}
=
\begin{bmatrix}
	27 & 0 & -22  \\
	10 & 12 & -28  \\
	16 & 0 & -11 
\end{bmatrix}
\end{equation}

Матрицы не являются перестановочными.

\begin{equation}
	A = 
\begin{vmatrix}
	5 & 3 & -7 \\
	-1 & 6 & -3 \\
	2 & -2 & 1
\end{vmatrix}
\underset{II + I}{=} 
\begin{vmatrix}
	5 & 8 & -7 \\
	-1 & 5 & -3 \\
	2 & 0 & 1
\end{vmatrix}
\underset{I - 2 * III}{=} 
\begin{vmatrix}
	19 & 8 & -7 \\
	5 & 5 & -3 \\
	0 & 0 & 1
\end{vmatrix}
\end{equation}
Теперь используем правила лапласа по 3 строке

\begin{equation}
\begin{vmatrix}
	19 & 8 & -7 \\
	5 & 5 & -3 \\
	0 & 0 & 1
\end{vmatrix}
=
19 * 5 - 5 * 8 = 55
\end{equation}


\end{document}
