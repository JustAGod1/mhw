\documentclass[titlepage]{article}
\usepackage{amsmath}
\usepackage{fancyhdr}
\usepackage{graphicx}
\usepackage[T1, T2A]{fontenc}
\usepackage[utf8]{inputenc}
\usepackage[russian, english]{babel}



\title{Линейная алгебра 13.09}
\date{\today}
\author{Юрий Баринов}

\pagestyle{fancy}
\lhead{Юрий Баринов}\chead{Линейная алгебра 13.09}\rhead{\today}

\newcounter{ProblemNum}

\newcommand{\problem}[1]{
\stepcounter{ProblemNum}
\newpage
\section*{Задача \theProblemNum. #1}
}

\newcommand{\solution}{
	\subsection*{Решение}
}

\begin{document}
\maketitle{}
\problem{Номер 59}
\begin{equation}
	\begin{vmatrix}
		5-x & 6 & -3 \\
		-1 & -x & 1 \\
		1 & 2 & 1-x \\
	\end{vmatrix}
	= 0
\end{equation}
\solution{}
Вычтем 3 строку из 2ой. Получим
\begin{equation}
	\begin{vmatrix}
		5-x & 6 & -3 \\
		0 & 2-x & 2-x \\
		1 & 2 & 1-x \\
	\end{vmatrix}
	= 0
\end{equation}
Теперь вычтем 3 столюец из 2го. Получим:
\begin{equation}
	\begin{vmatrix}
		5-x & 9 & -3 \\
		0 & 0 & 2-x \\
		1 & 1+x & 1-x \\
	\end{vmatrix}
	= 0
\end{equation}

Теперь воспользуемся формулой Лапласа и решим уравнение: 
\begin{align}
(x-2)((5-x)(1+x) - 9) = 0 \\
(x-2)(5 + 5x - x -x^2 - 9) = 0 \\
(x-2)(-x^2 + 4x - 4) = 0
\end{align}
Найдем решение уравнения $(-x^2 + 4x - 4) = 0$ 
\fontencoding{T1}
\begin{align}
	\emph{D} = 16 - 16 = 0 \\
	x_{1,2} = \frac{-4 \pm\sqrt{0}}{-2} \\
	x = 2	
\end{align}
\fontencoding{T2A}

Отсюда получаем итоговое уравнение


\begin{equation}
	(x-2)^2 = 0
\end{equation}

Таким образом решение уравнения следующее:

\begin{equation}
x = 2
\end{equation}

\problem{Номер 73}
Найти определитель матрицы$$
\begin{bmatrix}
	-7 & 3 & 8 \\
	4 & 2 & 3 \\
	3 & -5 & 4
\end{bmatrix}
$$
двумя способами

\solution
\subsection*{1 Вариант}
Вычтем из первой строки третью умноженную на два
\begin{equation}
\begin{bmatrix}
	-13 & 13 & 0 \\
	4 & 2 & 3 \\
	3 & -5 & 4
\end{bmatrix}
\end{equation}
Добавим 2 столбец к первому
\begin{equation}
\begin{bmatrix}
	0 & 13 & 0 \\
	6 & 2 & 3 \\
	-2 & -5 & 4
\end{bmatrix}
\end{equation}
Теперь используя формулу Лапласа получаем
\begin{equation}
\begin{vmatrix}
	0 & 13 & 0 \\
	6 & 2 & 3 \\
	-2 & -5 & 4
\end{vmatrix}
= -13 * (24 + 6) = -390
\end{equation}
\subsection*{2 Вариант}
\begin{equation}
\begin{vmatrix}
	-7 & 3 & 8 \\
	4 & 2 & 3 \\
	3 & -5 & 4
\end{vmatrix} = -56 + 27 - 160 - 48 - 48 - 105 = -390
\end{equation}


\problem{Номер 74}
Найти оперелитель матрицы
$$
\begin{bmatrix}
	2 & -9 & 6 \\
	5 & 4 & -2 \\
	-3 & -5 & 7
\end{bmatrix}
$$
двумя способами
\solution
\subsection*{1 Вариант}

Добавим к 3 строке 1 строку
\begin{equation}
\begin{bmatrix}
	2 & -9 & 6 \\
	5 & 4 & -2 \\
	-1 & -14 & 13
\end{bmatrix}
\end{equation}
Добавим ко второму столбцу первый
\begin{equation}
\begin{bmatrix}
	2 & -3 & 6 \\
	5 & 2 & -2 \\
	-1 & -1 & 13
\end{bmatrix}
\end{equation}
Вычтем из первого столбца второй
Добавим к 3 столбцу первый умноженный на 13
\begin{equation}
\begin{bmatrix}
	5 & -3 & -33 \\
	3 & 2 & 24 \\
	0 & -1 & 0
\end{bmatrix}
\end{equation}
Теперь применим правило Лапласа
\begin{equation}
\begin{vmatrix}
	5 & -3 & -33 \\
	3 & 2 & 24 \\
	0 & -1 & 0
\end{vmatrix} = 120 + 99 = 219
\end{equation}
\subsection*{2 Вариант}
\begin{equation}
\begin{vmatrix}
	5 & -3 & -33 \\
	3 & 2 & 24 \\
	0 & -1 & 0
\end{vmatrix} = 56 - 54 - 150 + 72 + 315 - 20 = 219
\end{equation}

\problem{Номер 118}
Решить выражение
\begin{equation}
	\begin{bmatrix}
		2 & -1 & -1 \\
		3 & 4 & -2
	\end{bmatrix}
	\begin{bmatrix}
		1 & -3 \\
		2 & 4 \\
		0 & 2
	\end{bmatrix}
	\begin{bmatrix}
		2 & -1 \\
		0 & 3
	\end{bmatrix}
\end{equation}
\solution
\begin{equation}
	\begin{bmatrix}
		2 & -1 & -1 \\
		3 & 4 & -2
	\end{bmatrix}
	\begin{bmatrix}
		1 & -3 \\
		2 & 4 \\
		0 & 2
	\end{bmatrix}
	=
	\begin{bmatrix}
		0 & -12 \\
		11 & -3
	\end{bmatrix}
\end{equation}
\begin{equation}
	\begin{bmatrix}
		0 & -12 \\
		11 & -3
	\end{bmatrix}
	\begin{bmatrix}
		2 & -1 \\
		0 & 3
	\end{bmatrix}
	=
	\begin{bmatrix}
		0 & -36 \\
		22 & -20
	\end{bmatrix}
\end{equation}

\problem{Номер 114}
Решить выражение
\begin{equation}
	\begin{bmatrix}
		1 & -4 & -2
	\end{bmatrix}
	\begin{bmatrix}
		2 \\
		0 \\
		-3
	\end{bmatrix}
\end{equation}
\solution
\begin{equation}
	\begin{bmatrix}
		1 & -4 & -2
	\end{bmatrix}
	\begin{bmatrix}
		2 \\
		0 \\
		-3
	\end{bmatrix}
	=
	\begin{bmatrix}
		8
	\end{bmatrix}
\end{equation}

\problem{Номер 115}
Решить выражение
\begin{equation}
	\begin{bmatrix}
		-2 \\
		3 \\
		1
	\end{bmatrix}
	\begin{bmatrix}
		4 & -1 & 3
	\end{bmatrix}
\end{equation}
\solution
\begin{equation}
	\begin{bmatrix}
		-2 \\
		3 \\
		1
	\end{bmatrix}
	\begin{bmatrix}
		4 & -1 & 3
	\end{bmatrix}
	=
	\begin{bmatrix}
		-8 & 2 & -6 \\
		12 & -3 & 9 \\
		4 & -1 & 3
	\end{bmatrix}
\end{equation}


\end{document}
